\documentclass[a4paper]{article}
\usepackage{graphicx}
\graphicspath{ {./images/} }
\usepackage{amsmath}
\usepackage{mathtools}
\usepackage[makeroom]{cancel}
\usepackage[utf8]{inputenc}
\usepackage{amssymb}
\title{Mästarprovet 13: Strömkretsen}
\author{Hassan Al Noori}
\usepackage[margin=0.5in]{geometry}
\begin{document}


\begin{Huge}


\begin{align*}
L = \frac{L_{0}}{1+I^{2}}
\end{align*} 

\begin{align*}
U = L\frac{dI}{dt}
\end{align*} 


\begin{align*}
I = -C\frac{dU}{dt}
\end{align*}

\begin{align*}
\frac{d^{2}I}{dt^{2}}= \frac{2I}{1+I^{2}}\left(\frac{dI}{dt}\right)^{2}-\frac{I(1+I^{2})}{L_{0}C}
\end{align*}
\begin{align*}
t=0, \quad I=0, \quad \frac{dI}{dt}=\frac{U_{0}}{L_{0}}
\end{align*}
\begin{align*}
\tilde{y}'(t) =
\begin{bmatrix}
y_{1}'(t) \\ 
y_{2}'(t) \\
\end{bmatrix}
= \begin{bmatrix}
y_{2}(t)\\
\frac{2y_{1}(t)y_{2}^{2}(t)}{1+y_{1}(t)^2}-\frac{1+y_{1}(t)^{2}}{L_{0}C}\\
\end{bmatrix}
\end{align*}


\begin{align*}
\frac{d^{2}I}{dt^{2}}=\frac{2I}{1+I^{2}}\left(\frac{dI}{dt}\right)^{2}-\frac{I\left(1+I^{2}\right)}{L_{0}C}
\end{align*}
\end{Huge}
\begin{equation*}
\begin{aligned}
\end{aligned}
\end{equation*}
\LARGE
\begin{align*}
E(t) & = U(t)^{2}-log(1+I(t)^{2})\\
\frac{dE}{dt} & = \frac{d}{dt}\left(U(t)^{2}-log(1+I(t)^{2})\right)\\
 & = \frac{d}{dt}\left(\left(\frac{L_{0}\frac{dI}{dt}}{1+I(t)^{2}})\right)^{2}-log\left(1+I(t)^{2}\right)\right) \\
& = \frac{d}{dt}\left(\frac{L_{0}\frac{dI}{dt}}{1+I(t)^{2}}\right)^{2}-\frac{d}{dt}\left(log\left(1+I(t)^{2}\right)\right)\\
& = ...\\
& = 2\left({\frac{\frac{dI}{dt}\left(1+I(t)^{2}\left(\frac{d^{2}I}{dt^{2}}\right)-2I(t)\left(\frac{dI}{dt}\right)^{2}\right)}{\left(1+I(t)^{2}\right)^{3}}} -   {\frac{I(t)\left(\frac{dI}{dt}\right)}{\left(1+I(t)^{2}\right)}}\right)\\
& = 2\left(\frac{\frac{dI}{dt}}{1+I(t)^{2}}\right)   \left({\frac{1+I(t)^{2}\left(\frac{d^{2}I}{dt^{2}}\right)-2I(t)\left(\frac{dI}{dt}\right)^{2}}{\left(1+I(t)^{2}\right)^{2}}} -   {\frac{I(t)}{1}}\right)\\
\frac{dE}{dt} & = 0 \quad \forall \quad t \in \mathbb{R} \quad \Longleftrightarrow \frac{d}{dt}I(0) = 0 \quad or \quad I(0)= 0 \quad \because \\
\frac{dE}{dt} &= 0 \Longrightarrow 
{\frac{d^{2}I}{dt^{2}}}=\left(\frac{2I(t)\frac{dI}{dt}}{1+I(t)^{2}}+\frac{4\left(\frac{dI}{dt}\right)^{3}I(t)}{\left(1+I(t)^{2}\right)^{3}}\right)\frac{\left(1+I(t)^{2}\right)^{2}}{2\left(\frac{dI}{dt}\right)}
\end{align*}\\
Which by definition is a second order autonomous ODE. Thus, If $I=c$ is a specific solution, then its phase line is indenpendent of the time at which initial conditions are applied. Hence
\begin{align*}
\frac{d^{2}I}{dt^{2}}=0,\quad I_{0}=0 \quad \therefore\quad\frac{dE}{dt}=0\\
\square
\end{align*}
Which is given by the fact that the capacitor and magnetic field had been fully charged at $t=0$.

\begin{align*}
\end{align*}
\begin{align*}
& = 2\left(\cancelto{0}{\frac{\frac{dI}{dt}\left(1+I(t)^{2}\left(\frac{d^{2}I}{dt^{2}}\right)-2I(t)\left(\frac{dI}{dt}\right)^{2}\right)}{\left(1+I(t)^{2}\right)^{3}}} +   \cancelto{0}{\frac{I(t)\left(\frac{dI}{dt}\right)}{\left(1+I(t)^{2}\right)}}\right)
\end{align*}\\

\begin{align*}
\tilde{y'}=
\begin{bmatrix}
\frac{dU}{dt}\\ \\ 
\frac{dI}{dt}\\
\end{bmatrix}
= \begin{bmatrix}
\frac{L_{0}\left(\frac{dI}{dt}\right)}{1+I^{2}}\\\\
-C\frac{dU}{dt}\\
\end{bmatrix}
\end{align*}







\end{document}

